\begin{Verbatim}[commandchars=\\\{\}]
function hirschIndex(\textcolor{cyan}{int}[] arr) 
    \textcolor{gray}{//wir fangen bei 0 an zu zählen}
   \textcolor{magenta}{if} arr.length == 0 \textcolor{magenta}{then}
        \textcolor{magenta}{return} 0
    \textcolor{magenta}{end}
    \textcolor{gray}{//aus vorlesung}
   mergeSort(arr)
    \textcolor{gray}{//array um zitieurngen zu zählen}
   \textcolor{cyan}{int} cite[] = new \textcolor{cyan}{int}[arr.length]
    cite[0] = 1
    \textcolor{cyan}{int} j = 0
    \textcolor{magenta}{for} \textcolor{cyan}{int} i = 1, i < arr.length, i++ \textcolor{magenta}{do}
        \textcolor{gray}{//wenn wir werk wechseln, zählen wir im nächsten eintrag weiter}
       \textcolor{magenta}{if} arr[i - 1] != arr[i] \textcolor{magenta}{then}
            j++
        \textcolor{magenta}{end}
        \textcolor{gray}{//momentane zitierung ikrementieren}
       cite[j]++
    \textcolor{magenta}{end}
    \textcolor{gray}{//aus vorlesung}
   mergeSort(cite)
    \textcolor{gray}{//hirsch bestimmen und zurückgeben}
   \textcolor{cyan}{int} hirsch = 0
    \textcolor{magenta}{for} \textcolor{cyan}{int} i = 0, i < cite.length, i++ \textcolor{magenta}{do}
        \textcolor{gray}{//mit min prüfen wir beide Bedingungen des Hirschindex}
       \textcolor{gray}{//mit max nehmen wir den neune gültigen wert falls er größer ist als Vorher}
       hirsch = Math.max(Math.min(cite[i], cite.length - i), hirsch)
    \textcolor{magenta}{end}
    \textcolor{magenta}{return} hirsch
\textcolor{magenta}{end}
\end{Verbatim}
